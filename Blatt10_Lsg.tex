\documentclass[a4paper,10pt]{article}
\usepackage[ngerman]{babel}		%dt. Übersetzung und Umlaute
\usepackage[utf8]{inputenc}		%Umlaute direkt eingeben
\usepackage{mathtools}			%Mathekrams
\usepackage{paralist}			%bessere Listen
\usepackage{amssymb}			%Mathesymbole
\usepackage{amsthm}			%typesetting theorems (Text über = u.ä.)
\usepackage{fancyhdr} 			%Headerstyles
\usepackage{verbatim}			%Sourcecode einfügen
\usepackage[margin=2.0cm,headheight=40pt,top=3cm]{geometry}
\usepackage{tikz}
\usepackage{cancel}
\usepackage{stmaryrd}
\usepackage{colortbl}
\usetikzlibrary{matrix,positioning,arrows, automata}

\pagestyle{fancy}
\renewcommand{\headrulewidth}{0.4pt}
\renewcommand{\footrulewidth}{0.4pt}
\lhead{Blatt 10}
\rhead{}
\cfoot{}
\rfoot{\thepage}
\begin{document}
	\parindent0pt
	\textbf{Aufgabe 01}
	\begin{compactenum} [(a)]
		\item Es existiert eine Gewinnstrategie für Spoiler bei einem 4-Runden EF Spiel auf $ \mathcal{A} $ und $ \mathcal{B} $. Diese lässt sich leicht aus der Formel $ \psi $ aus Aufgabe (b) ablesen:\\
		\begin{tabbing}
			\underline{Runde 1:} \= Spoiler wählt ein $ b_1 \in B $ genau so, dass gilt:\\
			\> $ \mathcal{B} \models \big(\exists y \exists z \forall v (((E(x,v)\vee E(v,x))\vee  (E(y,v)\vee E(v,y)))\vee (E(z,v)\vee E(v,z)))\big)[b_1] $ \\
			\> $ b_1 $ ist genau der linke Knoten der Raute in $ \mathcal{B} $ \\
			\> Duplicator wählt nun ein beliebigen Knoten $ a_1 \in A$, so dass gilt:\\
			\> $ \mathcal{A} \models \big(\forall y \forall z \exists v (((\lnot E(x,v)\wedge \lnot E(v,x))\wedge  (\lnot E(y,v)\wedge \lnot E(v,y)))\wedge (\lnot E(z,v)\wedge \lnot E(v,z)))\big) [a_1] $ \\
			
			\underline{Runde 2:} \> Spoiler wählt ein $ b_2 \in B $ genau so, dass gilt:\\
			\> $ \mathcal{B} \models \big(\exists z \forall v (((E(x,v)\vee E(v,x))\vee  (E(y,v)\vee E(v,y)))\vee (E(z,v)\vee E(v,z)))\big)[b_1,b_2] $ \\
			\> $ b_2 $ ist genau der recht Knoten der Rauten in $ \mathcal{B} $\\
			\> Duplicator wählt erneut einen beliebigen Knoten $ a_2 \in A$ so dass gilt:\\
			\> $ \mathcal{A} \models \big(\forall z \exists v (((\lnot E(x,v)\wedge \lnot E(v,x))\wedge  (\lnot E(y,v)\wedge \lnot E(v,y)))\wedge (\lnot E(z,v)\wedge \lnot E(v,z)))\big) [a_1,a_2] $ \\
			\underline{Runde 3:} \> Spoiler wählt ein $ b_3 \in B $ genau so, dass gilt:\\
			\> $ \mathcal{B} \models \big(\forall v (((E(x,v)\vee E(v,x))\vee  (E(y,v)\vee E(v,y)))\vee (E(z,v)\vee E(v,z)))\big)[b_1,b_2,b_3] $ \\
			\> $ b_3 $ ist genau der oberste linke Knoten in $ \mathcal{B} $\\
			\> Duplicator wählt wieder einen beliebigen Knoten $ a_3 \in A$ so dass gilt:\\
			\> $ \mathcal{A} \models \big(\exists v (((\lnot E(x,v)\wedge \lnot E(v,x))\wedge  (\lnot E(y,v)\wedge \lnot E(v,y)))\wedge (\lnot E(z,v)\wedge \lnot E(v,z)))\big) [a_1,a_2,a_3] $ \\
			\underline{Runde 4:} \> Spoiler wählt einen Knoten $ a_4 \in A$ genau so, dass gilt:\\
			\> $ \mathcal{A} \models \big(((\lnot E(x,v)\wedge \lnot E(v,x))\wedge  (\lnot E(y,v)\wedge \lnot E(v,y)))\wedge (\lnot E(z,v)\wedge \lnot E(v,z))\big) [a_1,a_2,a_3,a_4] $ \\
			\> Der Knoten $ a_4 $ darf also weder $ a_1, a_2 $ noch $ a_3 $ als Nachbarn haben.\\
			\> Duplicator wählt nun einen beliebigen Knoten $ b_4 \in B $, so dass gilt:\\
			\> $ \mathcal{B} \models \big(((E(x,v)\vee E(v,x))\vee  (E(y,v)\vee E(v,y)))\vee (E(z,v)\vee E(v,z))\big)[b_1,b_2,b_3,b_4] $ \\
			\> $ b_4 $ ist nun gezwungener maßen entweder mit $ b_1, b_2 $ oder $ b_3 $ benachbart und verletzt somit die Formel $ \psi $\\
		\end{tabbing}
		
		Gewinnstrategie für Spoiler für m = 3:\\
		\\
		TODO
		\\
		
		
		\item $ \psi := \forall x \forall y \forall z \exists v (((\lnot E(x,v)\wedge \lnot E(v,x))\wedge  (\lnot E(y,v)\wedge \lnot E(v,y)))\wedge (\lnot E(z,v)\wedge \lnot E(v,z))) $\\
		$ \lnot \psi := \exists x \exists y \exists z \forall v (((E(x,v)\vee E(v,x))\vee  (E(y,v)\vee E(v,y)))\vee (E(z,v)\vee E(v,z))) $ \\
		$ \mathcal{A} \models \psi $ und $ \mathcal{B} \not\models \psi$
	\end{compactenum}\ \\
	
	\textbf{Aufgabe 02}\\
	TODO
	\\
	
	\textbf{Aufgabe 03}\\
	\begin{compactenum} [(a)]
		\item \begin{compactenum} [(i)]
			\item TODO
			\item TODO
		\end{compactenum}
		\item $ \varphi(x,z) := \exists y \Big(E(z,y)\rightarrow \big(\forall y E(x,y) \wedge \lnot \exists x E(x,y)\big)\Big) $ \\
		\begin{compactenum} [(i)]
			\item Siehe Aufgabenteil (b)
			\\
			\item \begin{tabbing}
				$ \varphi(x,z) $\= $ \equiv \exists y \Big(\lnot E(z,y)\vee \big(\forall y E(x,y) \wedge \lnot \exists x E(x,y)\big)\Big) \qquad \qquad$ \= Elimination von "$ \rightarrow $"\\
				\> $ \equiv \exists y \Big(\lnot E(z,y)\vee \big(\forall y E(x,y) \wedge \forall x \lnot E(x,y)\big)\Big)$ \> $ \lnot \exists x \psi \equiv \forall x \lnot \psi $\\
				\> $ \equiv \exists y \Big(\lnot E(z,y)\vee \big(\forall z_1 E(x,z_1) \wedge \forall z_2 \lnot E(z_2,y)\big)\Big)$ \> Umbenennung von gebundenen Variablen\\
				\> $ \equiv \exists y \Big(\lnot E(z,y)\vee \forall z_1 \forall z_2 \big( E(x,z_1) \wedge \lnot E(z_2,y)\big)\Big)$ \> Zusammenlegung der Konjunktion\\
				\> $ \equiv \exists y \forall z_1 \forall z_2 \Big(\lnot E(z,y)\vee \big( E(x,z_1) \wedge \lnot E(z_2,y)\big)\Big)$ \> Zusammenlegung der Disjunktion\\
			\end{tabbing}
			Diese Formel ist in PNF und NNF
		\end{compactenum}
		\item TODO
	\end{compactenum}
	\textbf{Aufgabe 04}\\
	\begin{verbatim}
		:- module(pure_literal, [ knf_shell/0, pure_literal/2 ]).
		:- use_module(al_nf, [ al2menge/2 ]).
		:- use_module(al_literals, [ negate_lit/2]).
		
		knf_shell :-
		read(current_input, T),
		write_menge(T).
		
		write_menge(bye) :- !.
		write_menge(T) :-
		al2menge(T, TM),
		write(TM), nl,
		knf_shell.
		
		is_literal(L, KM) :-
		member(K, KM), member(L, K).
		
		is_pure_literal(L, KM) :-
		is_literal(L, KM),
		negate_lit(L, LN),
		\+ is_literal(LN, KM).
		
		remove_literal(L, KM, KM2) :-
		findall(K, (member(K, KM), \+ member(L, K)), KM2).
		
		pure_literal(KM, KM2) :-
		is_pure_literal(L, KM), !,
		remove_literal(L, KM, KM3),
		pure_literal(KM3, KM2).
		pure_literal(KM, KM).
	\end{verbatim}
\end{document}
