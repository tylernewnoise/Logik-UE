\documentclass[a4paper,10pt]{article}
\usepackage[ngerman]{babel}		%dt. Übersetzung und Umlaute
\usepackage[utf8]{inputenc}		%Umlaute direkt eingeben
\usepackage{mathtools}			%Mathekrams
\usepackage{paralist}			%bessere Listen
\usepackage{amssymb}			%Mathesymbole
\usepackage{amsthm}			%typesetting theorems (Text über = u.ä.)
\usepackage{fancyhdr} 			%Headerstyles
\usepackage{verbatim}			%Sourcecode einfügen
\usepackage[margin=2.0cm,headheight=40pt,top=3cm]{geometry}
\usepackage{tikz}
\usepackage{tikz-qtree}
\usepackage{cancel}
\usepackage{stmaryrd}
\usepackage{colortbl}
\usetikzlibrary{matrix,positioning,arrows, automata}

\pagestyle{fancy}
\renewcommand{\headrulewidth}{0.4pt}
\renewcommand{\footrulewidth}{0.4pt}
\lhead{Jonas Panizza (567687)\\
	Fabian Bucher (577513)\\
	Falko Becker (559053)\\
	Blatt 04}
\rhead{ÜG Di 13-15 Jens Keppeler\\
	ÜG Di 13-15 Jens Keppeler \\
	ÜG Di 13-15 Jens Keppeler \\
	Rückgabe ÜG Di 13-15 Jens Keppeler}
\cfoot{}
\rfoot{\thepage}
\begin{document}
	\parindent0pt
	\textbf{Aufgabe 01}
	\begin{compactenum} [(a)]
		\item $ \varphi_1 = ((R\vee S) \wedge (\lnot R \vee S \vee T) \wedge (\lnot S \vee \lnot T)\wedge (R \vee \lnot T) \wedge (T\vee \lnot S \vee \lnot R \vee U) \wedge (\lnot S \vee \lnot U)) $
		\item \begin{compactitem}
			\item $ \mathcal{I} \models \Gamma_1 $ mit $ \mathcal{I}(R) = 1;\ \mathcal{I}(S) = 0;\ \mathcal{I}(T) = 1;\ \mathcal{I}(U) = 0$
			\item Für alle Interpretationen I gilt: $ \mathcal{I}\not\models\Gamma_2 $ \\ \\
			Resolutionswiederlegung:
			\begin{tabbing}
				\= \qquad \= \qquad\qquad\qquad \= \qquad\qquad \kill
				\>(1) \> $ \{R, \lnot Q\} $ \> in $ \Gamma_2 $ \\
				\>(2) \> $ \{\lnot R, \lnot Q\} $ \> in $ \Gamma_2 $ \\
				\>(3) \> $ \{\lnot Q \} $ \> Resolution von (1) und (2) \\
				\>(4) \> $ \{Q ,\lnot R\} $ \> in $ \Gamma_2 $\\
				\>(5) \> $ \{\lnot R\} $ \> Resolution von (3) und (4) \\
				\>(6) \> $ \{Q, R, \lnot S\} $ \> in $ \Gamma_2 $\\
				\>(7) \> $ \{Q, \lnot S\} $ \> Resolution von (5) und (6) \\
				\>(8) \> $ \{\lnot S\} $ \> Resolution von (3) und (7) \\
				\>(9) \> $ \{Q, R, S\} $ \> in $ \Gamma_2 $ \\
				\>(10) \> $ \{Q, R\} $ \> Resolution von (8) und (9) \\
				\>(11) \> $ \{Q\} $ \> Resolution von (5) und (10)\\
				\>(12) \> $ \ \varnothing $ \> Resolution von (3) und (11) \\
			\end{tabbing}
			\item Für alle Interpretationen I gilt: $ \mathcal{I}\not\models\Gamma_3 $ \\ \\
			Resolutionswiederlegung:
			
			\begin{tabbing}
				\= \qquad \= \qquad\qquad\qquad \= \qquad\qquad \kill
				\>(1) \> $ \{R\} $ \> in $ \Gamma_3 $ \\
				\>(2) \> $ \{\lnot R, \lnot Q\} $ \> in $ \Gamma_3 $ \\
				\>(3) \> $ \{\lnot Q \} $ \> Resolution von (1) und (2) \\
				\>(4) \> $ \{P ,\lnot R, S\} $ \> in $ \Gamma_3 $\\
				\>(5) \> $ \{\lnot P, S\} $ \> in $ \Gamma_3 $ \\
				\>(6) \> $ \{\lnot R, S\} $ \> Resolution von (4) und (5) \\
				\>(7) \> $ \{S\} $ \> Resolution von (1) und (6) \\
				\>(8) \> $ \{\lnot S, Q\} $ \> in $ \Gamma_3 $ \\
				\>(9) \> $ \{Q\} $ \> Resolution von (7) und (8) \\
				\>(10) \> $ \{\varnothing\} $ \> Resolution von (3) und (9) \\
			\end{tabbing}
		\end{compactitem}
	\end{compactenum}
	\textbf{Aufgabe 02} \\
	\underline{Schritt 1:} Wir listen alle Subformeln von $ \varphi  $ auf, die keine Literale sind: \\
	
	$ \varphi := (\underbrace{(\underbrace{(P\vee \lnot Q)}_{\psi_3}\wedge S)}_{\psi_1} \rightarrow \underbrace{\lnot \underbrace{(Q \vee \lnot S)}_{\psi_4}}_{\psi_2}) $ \\
	\begin{tabbing}
		$ \varphi' :=$ 
		\= $ \quad X_p $ \qquad \qquad \qquad \qquad \qquad \qquad \qquad \= \\
		\> $ \wedge (X_\varphi \leftrightarrow (X_{\psi_1}\rightarrow X_{\psi_2})$ \> (da $ \varphi = (\psi_1 \rightarrow \psi_2) $) \\
		\> $ \wedge (X_{\psi_1} \leftrightarrow (X_{\psi_3} \wedge S)) $ \> (da $ \psi_1 = (\psi_3 \wedge S)) $ \\
		\> $ \wedge (X_{\psi_2} \leftrightarrow (\lnot X_{\psi_4})) $ \> (da $ \psi_2 = \lnot \psi_4 $) \\
		\> $ \wedge (X_{\psi_3} \leftrightarrow (P \vee \lnot Q)) $ \> (da $ \psi_3 = (P \vee \lnot Q) $) \\
		\> $ \wedge (X_{\psi_4} \leftrightarrow (Q \vee \lnot S)) $ \> (da $ \psi_4 = (Q \vee \lnot S) $)
	\end{tabbing}
	Es gilt: $ \varphi $ ist erfüllbar $ \Longleftrightarrow  \varphi' $ ist erfüllbar \\
	\newpage
	\underline{Schritt 2:} Umwandlung von $ \varphi' $ in eine äquivalente KNF Formel: \\
	\begin{tabbing}
		$ \varphi_K := $
		\= $ \quad X_\varphi $ \\
		\> $ \wedge (\lnot X_\varphi \vee \lnot X_{\psi_1} \vee X_{\psi_2}) \wedge (X_\varphi \vee X_{\psi_1}) \wedge (X_\varphi \vee \lnot X_{\psi_2}) $ \\
		\> $ \wedge (\lnot X_{\psi_1} \vee X_{\psi_3}) \wedge (\lnot X_{\psi_1} \vee S) \wedge (\lnot X_{\psi_3} \vee \lnot S \vee X_{\psi_1}) $ \\
		\> $ \wedge (\lnot X_{\psi_2} \vee \lnot X_{\psi_4}) \wedge (X_{\psi_4} \vee X_{\psi_2}) $ \\
		\> $ \wedge (\lnot X_{\psi_3} \vee P \vee \lnot Q) \wedge (\lnot P \vee X_{\psi_3}) \wedge (Q \vee X_{\psi_3}) $ \\
		\> $ \wedge (\lnot X_{\psi_4} \vee Q \vee \lnot S) \wedge (\lnot Q \vee X_{\psi_4}) \wedge (S \vee X_{\psi_4}) $
	\end{tabbing}
	Da $ \varphi_K $ äquivalent zu $ \varphi' $ und $ \varphi' $ erfüllbarkeitsäquivalent zu $ \varphi $ ist, ist insgesamt $ \varphi_K $\\
	erfüllbarkeitsäquivalent zu $ \varphi $.\\ \\
	
	\textbf{Aufgabe 03}
	\begin{compactenum} [(a)]
		\item Sei $ i,j \in \{1,...,n\}; \quad t,t' \in K; \quad n \in \mathbb{N}_{>0} $\\
		$ \varphi_n := \bigwedge\limits_{i = 1}^n \bigwedge\limits_{j = 1}^n A_{i,j}^t \wedge \bigwedge\limits_{i=1}^{n-1} \bigwedge\limits_{j=1}^n (\bigwedge\limits_{(t,t') \not\in H} \lnot (A_{i,j}^t \wedge A_{i+1,j}^{t'}) \wedge (\bigwedge\limits_{(t,t') \not\in V} \lnot (A_{j,i}^t \wedge A_{j,i+1}^{t'})) $ \\
		$ \Gamma_n := \{\varphi_n\} $
		
		\item TODO
	\end{compactenum}
\end{document}
