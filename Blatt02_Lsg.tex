\documentclass[a4paper,10pt]{article}
\usepackage[ngerman]{babel}		%dt. Übersetzung und Umlaute
\usepackage[utf8]{inputenc}		%Umlaute direkt eingeben
\usepackage{mathtools}			%Mathekrams
\usepackage{paralist}			%bessere Listen
\usepackage{amssymb}			%Mathesymbole
\usepackage{amsthm}			%typesetting theorems (Text über = u.ä.)
\usepackage{fancyhdr} 			%Headerstyles
\usepackage{verbatim}			%Sourcecode einfügen
\usepackage[margin=2.5cm,headheight=40pt,top=3cm]{geometry}
\usepackage{tikz}
\usepackage{cancel}
\usepackage{stmaryrd}
\usepackage{colortbl}
\usetikzlibrary{matrix,positioning,arrows, automata}

\pagestyle{fancy}
\renewcommand{\headrulewidth}{0.4pt}
\renewcommand{\footrulewidth}{0.4pt}
\lhead{\textbf{Blatt 02}}
\rhead{}
\cfoot{}
\rfoot{\thepage}
\begin{document}
	\parindent0pt
	\textbf{Aufgabe 01}
	\begin{compactenum} [(a)]
		\item  \begin{tabbing} $
			\varphi_1 := \bigwedge\limits_{i,j=0}^n $
			\= $ ((A_{i,j}\wedge \lnot O_{i,j} \wedge \lnot G_{i,j}\wedge \lnot K_{i,j})\vee(\lnot A_{i,j}\wedge O_{i,j}\wedge\lnot G_{i,j}\wedge \lnot K_{i,j}) $ \\
			\> $ \vee ( \lnot A_{i,j}\wedge \lnot O_{i,j}\wedge G_{i,j}\wedge \lnot K_{i,j})\vee( \lnot A_{i,j}\wedge \lnot O_{i,j}\wedge \lnot G_{i,j}\wedge K_{i,j}))$
		\end{tabbing}\
		\item $ \varphi_2 $ erfüllt, wenn es mindestens ein Grundstück in der Stadt gibt, dessen Bewohner ein Katzenliebhaber ist und dessen vier Nachbarn ebenfalls Katzenliebhaber sind.\\
		\item $ \varphi_3 = \bigwedge\limits_{i,j \in \{2,..,29\}} \lnot G_{i,j} $\\
		\item $ \varphi_4 = \bigwedge\limits_{i,j \in \{2,...,29\}} (A_{i,j} \rightarrow ((O_{i-1,j} \vee O_{i+1,j} \vee O_{i,j-1} \vee O_{i,j+1}) \wedge (\lnot K_{i-1,j} \wedge \lnot K_{i+1,j} \wedge \lnot K_{i,j-1} \wedge \lnot K_{i,j+1})) $
	\end{compactenum}\ \\
	\textbf{Aufgabe 02}
	\begin{compactenum} [(a)]
		\item
		\begin{itemize}
			\item $ \varphi_1 \not\equiv \varphi_2$, da $ \llbracket \varphi_1 \rrbracket ^I = 0 \neq \llbracket \varphi_2 \rrbracket ^I = 1 \qquad$ mit $ I(A_0) = 1 $; $ I(A_1) = 1 $ $ (A_2) = 0 $
			\item $ \varphi_3 \not\equiv \varphi_4$, da $ \llbracket \varphi_3 \rrbracket ^I = 1 \neq \llbracket \varphi_4 \rrbracket ^I = 0 \qquad$ mit $ I(A_0) = 0 $; $ I(A_1) = 1 $
		\end{itemize}\
		\item Seien $I = \{i_1, \dots , i_m\}$,  $J = \{j_1, \dots , j_n\} $ endliche Mengen und sei f"ur jedes $i \in I$ und $j \in J$ ein Formel $\varphi_{i, j}$ gegeben.\\
		$\bigwedge\limits_{i \in I} \bigvee\limits_{j \in J} \varphi_{i, j} \equiv \bigvee\limits_{j \in J} \bigwedge\limits_{i \in I} \varphi_{i, j}$ f.a. $i \in I, j\in J$\\
		Es gilt nicht, dass in jedem Fall die Bedingung erf"ullt ist.
		\begin{proof}\ \\
			Fall $n = m = 2$\\
			$\bigwedge\limits_{i = 1}^2 \bigvee\limits_{j = 1}^2 \varphi_{i, j} = (\varphi_{1,1} \vee \varphi_{1,2}) \wedge (\varphi_{2,1} \vee \varphi_{2,2}) =: \psi$\\
		    $\bigvee\limits_{j = 1}^2 \bigwedge\limits_{i = 1}^2  \varphi_{i, j} = (\varphi_{1,1} \wedge \varphi_{2,1}) \vee (\varphi_{1,2} \wedge \varphi_{2,2}) =: \chi$\\\\
		    Wahrheitstabelle:\\ \\
		   \begin{tabular}{cccc|c|c|c|c|c|c}
		   	$\varphi_{1,1}$ & $\varphi_{1,2}$ & $\varphi_{2,1}$ & $\varphi_{2,2}$ & $(\varphi_{1,1} \vee \varphi_{1,2})$ & $(\varphi_{2,1} \vee \varphi_{2,2})$ & $(\varphi_{1,1} \wedge \varphi_{2,1})$ & $(\varphi_{1,2} \vee \varphi_{2,2})$ & $\psi$ & $\chi$ \\ \hline
		   	0 & 0 & 0 & 0 & 0 & 0 & 0 & TODO & 0 & 0\\
		   	0 & 0 & 0 & 1 & 0 & 1 & 0 &  & 0 & 0\\
		   	0 & 0 & 1 & 0 & 0 & 1 & 0 &  & 0 & 0\\
		   	0 & 0 & 1 & 1 & 0 & 1 & 0 &  & 0 & 1\\
		   	&&&&&&&&&\\
		   	0 & 1 & 0 & 0 & 1 & 0 & 0 &  & 0 & 0\\
		   	0 & 1 & 0 & 1 & 1 & 1 & 0 &  & 1 & 0\\
		   	0 & 1 & 1 & 0 & 1 & 1 & 0 &  & 1 & 0\\
		   	0 & 1 & 1 & 1 & 1 & 1 & 0 &  & 1 & 1\\
		   	&&&&&&&&&\\
		   	1 & 0 & 0 & 0 & 1 & 0 & 0 &  & 0 & 0\\
		   	1 & 0 & 0 & 1 & 1 & 1 & 0 &  & 1 & 0\\
		   	1 & 0 & 1 & 0 & 1 & 1 & 1 &  & 1 & 0\\
		   	1 & 0 & 1 & 1 & 1 & 1 & 1 &  & 1 & 1\\
		   	&&&&&&&&&\\
		   	1 & 1 & 0 & 0 & 1 & 0 & 0 &  & 0 & 1\\
		   	1 & 1 & 0 & 1 & 1 & 1 & 0 &  & 1 & 1\\
		   	1 & 1 & 1 & 0 & 1 & 1 & 1 &  & 1 & 1\\
		   	1 & 1 & 1 & 1 & 1 & 1 & 1 &  & 1 & 1\\
		   \end{tabular}\\
			Damit ist $\psi \ \cancel{\equiv} \ \chi$
		\end{proof}
	\end{compactenum} \

	\newpage
	\textbf{Aufgabe 3}
	\begin{compactenum} [(a)]
		\item 
		\begin{compactenum} [(i)]
			\item $ A_{42} $
			\item $ (A_1 \vee \lnot (0 \wedge 1)) $
			\item $ \lnot (0 \wedge A_2) \vee ((\lnot 1 \vee A_5) \wedge (A_3 \vee \lnot ((A_3 \vee \lnot 0) \vee \lnot A_4))) $
		\end{compactenum}\ \\
		\item $ \varphi $ ist unerfüllbar $\Leftrightarrow \varphi \equiv 0$
		$ \xLeftrightarrow{\text{Dualitätssatz}} \tilde{\varphi} \equiv \tilde{0} \Leftrightarrow \tilde{\varphi} \equiv 1 \xLeftrightarrow{\text{Lemma 2.24b}} \tilde{\varphi} $ ist allgemein gültig.\\
		\item 2-stelligen Junktor $\tilde{\rightarrow}$, so dass f"ur alle $X, Y \in \mathsf{AS}$ und alle Interpretationen $\mathcal{I}$ gilt: \[\llbracket X \tilde{\rightarrow} Y \rrbracket^\mathcal{I} = 1 - \llbracket X \rightarrow Y \rrbracket^{\tilde{\mathcal{I}}} \]
		Der Junktor $\tilde{\rightarrow}$ sei definiert durh:\\\\
		\begin{tabular}{c|c|c|c|c}
			$X$ & $Y$ & $\llbracket X \tilde{\rightarrow} Y \rrbracket^\mathcal{I}$ & $\llbracket X \tilde{\rightarrow} Y \rrbracket^{\tilde{\mathcal{I}}}$ &  $1 - \llbracket X \rightarrow Y \rrbracket^{\tilde{\mathcal{I}}}$\\ 
			\hline 
			0 & 0 & 1 & 1 & 0 \\ 
			\hline 
			0 & 1 & 1 & 0 & 1 \\ 
			\hline 
			1 & 0 & 0 & 1 & 0 \\ 
			\hline 
			1 & 1 & 1 & 1 & 0 \\ 
		\end{tabular} \ \\\\
		\textbf{Satz 2.28 (Dualit"atssatz der Aussagenlogik)}\\
		F"ur alle Formeln $\varphi, \chi \in \mathsf{AF}$, in denen keine Implikation vorkommt, gilt: $\varphi \equiv \chi \Leftrightarrow \tilde{\varphi} \equiv \tilde{\chi}$.\\
		Nach der Definition von $\tilde{\rightarrow}$, wollen wir wissen ob nun der Dualit"atssatz auch f"ur aussagenlogische Formeln mit Implikationen formuliert werden kann. \\
		Gem"a"s der Induktionsannahme gilt: $\llbracket \tilde{\varphi} \rrbracket^\mathcal{I} = 1 - \llbracket \varphi \rrbracket^{\tilde{\mathcal{I}}}$ und $\llbracket \tilde{\psi} \rrbracket^\mathcal{I} = 1 - \llbracket \psi \rrbracket^{\tilde{\mathcal{I}}}$ f"ur $\mathcal{I}$ eine beliebige Interpretation und f"ur $\varphi, \psi$ beliebige Formeln von $\mathsf{AL}$.\\
		Es ist nun zu schauen, ob auch $\llbracket \widetilde{\varphi \rightarrow \psi} \rrbracket^\mathcal{I} = 1 - \llbracket \varphi \rightarrow \psi \rrbracket^{\tilde{\mathcal{I}}}$ gilt.\\
		Per Definition ist $\widetilde{\varphi \rightarrow \psi} = \tilde{\varphi} \tilde{\rightarrow} \tilde{\psi}$. Zur "Uberpr"ufung ob $\llbracket \tilde{\varphi} \tilde{\rightarrow} \tilde{\psi} \rrbracket^{\mathcal{I}} = 1 - \llbracket \varphi \rightarrow \psi \rrbracket^{\tilde{\mathcal{I}}}$ gilt, schauen wir uns die Wahrheitstabelle an:\\\\
		\begin{tabular}{c|c|c||c|c|c|c}
			$\llbracket \tilde{\varphi} \rrbracket^{\mathcal{I}}$ & $\llbracket \tilde{\psi} \rrbracket^{\mathcal{I}}$ & $\llbracket \tilde{\varphi} \tilde{\rightarrow} \tilde{\psi} \rrbracket^{\mathcal{I}}$ &  $\llbracket \varphi \rrbracket^{\tilde{\mathcal{I}}}$ & $\llbracket \psi \rrbracket^{\tilde{\mathcal{I}}}$ & $\llbracket \varphi \rightarrow \psi \rrbracket^{\tilde{\mathcal{I}}}$ & $1 - \llbracket \varphi \rightarrow \psi \rrbracket^{\tilde{\mathcal{I}}}$ \\ 
			\hline 
			0 & 0 & 0 & 0 & 0 & 1 & 0 \\ 
			0 & 1 & 0 & 0 & 1 & 1 & 0 \\ 
			1 & 0 & 1 & 1 & 0 & 0 & 1 \\ 
			1 & 1 & 0 & 1 & 1 & 1 & 0 \\ 
		\end{tabular} \ \\\\
		Die Tabelle ist f"ur $\llbracket \tilde{\varphi} \tilde{\rightarrow} \tilde{\psi} \rrbracket^{\mathcal{I}} = 1 - \llbracket \varphi \rightarrow \psi \rrbracket^{\tilde{\mathcal{I}}}$ g"ultig und somit kann man mit der Definition von $\tilde{\rightarrow}$ den Qualit"atssatz auch f"ur aussagenlogische Formeln mit Implikationen formulieren.
	\end{compactenum}\
	
	\textbf{Aufgabe 4}
	\begin{compactenum} [(a)]
		\item Todo
		
		\item Todo
	\end{compactenum}
\end{document}
