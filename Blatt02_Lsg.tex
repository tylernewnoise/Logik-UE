\documentclass[a4paper,10pt]{article}
\usepackage[ngerman]{babel}		%dt. Übersetzung und Umlaute
\usepackage[utf8]{inputenc}		%Umlaute direkt eingeben
\usepackage{mathtools}			%Mathekrams
\usepackage{paralist}			%bessere Listen
\usepackage{amssymb}			%Mathesymbole
\usepackage{amsthm}			%typesetting theorems (Text über = u.ä.)
\usepackage{fancyhdr} 			%Headerstyles
\usepackage{verbatim}			%Sourcecode einfügen
\usepackage[margin=2.5cm,headheight=40pt,top=3cm]{geometry}
\usepackage{tikz}
\usepackage{cancel}
\usepackage{stmaryrd}
\usepackage{colortbl}
\usetikzlibrary{matrix,positioning,arrows, automata}

\pagestyle{fancy}
\renewcommand{\headrulewidth}{0.4pt}
\renewcommand{\footrulewidth}{0.4pt}
\lhead{Jonas Panizza (567678)\\
	Fabian Bucher (577513)\\
	Falko Becker (559053)\\
	Blatt 02}
\rhead{ÜG Di 13-15 Jens Keppeler\\
	ÜG Di 13-15 Jens Keppeler \\
	ÜG Di 13-15 Jens Keppeler \\
	Rückgabe ÜG Di 13-15 Jens Keppeler}
\cfoot{}
\rfoot{\thepage}
\begin{document}
	\parindent0pt
	\textbf{Aufgabe 01}
	\begin{compactenum} [(a)]
		\item  \begin{tabbing} $
			\varphi_1 := \bigwedge\limits_{i,j=0}^n $
			\= $ ((A_{i,j}\wedge \lnot O_{i,j} \wedge \lnot G_{i,j}\wedge \lnot K_{i,j})\vee(\lnot A_{i,j}\wedge O_{i,j}\wedge\lnot G_{i,j}\wedge \lnot K_{i,j}) $ \\
			\> $ \vee ( \lnot A_{i,j}\wedge \lnot O_{i,j}\wedge G_{i,j}\wedge \lnot K_{i,j})\vee( \lnot A_{i,j}\wedge \lnot O_{i,j}\wedge \lnot G_{i,j}\wedge K_{i,j}))$
		\end{tabbing} 
		
		 
		\item $ \varphi_2 $ erfüllt, wenn es mindestens ein Grundstück in der Stadt gibt, dessen Bewohner ein Katzenliebhaber ist und dessen vier Nachbarn ebenfalls Katzenliebhaber sind.
		\item $ \varphi_3 = \bigwedge\limits_{i,j \in \{2,..,29\}} \lnot G_{i,j} $
		\item $ \varphi_4 = \bigwedge\limits_{i,j \in \{2,...,29\}} (A_{i,j} \rightarrow ((O_{i-1,j} \vee O_{i+1,j} \vee O_{i,j-1} \vee O_{i,j+1}) \wedge (\lnot K_{i-1,j} \wedge \lnot K_{i+1,j} \wedge \lnot K_{i,j-1} \wedge \lnot K_{i,j+1})) $
	\end{compactenum} \
	
	\textbf{Aufgabe 02}
	\begin{compactenum} [(a)]
		\item
		\begin{itemize}
			\item $ \varphi_1 \not\equiv \varphi_2$, da $ \llbracket \varphi_1 \rrbracket ^I = 0 \neq \llbracket \varphi_2 \rrbracket ^I = 1 \qquad$ mit $ I(A_0) = 1 $; $ I(A_1) = 1 $ $ (A_2) = 0 $
			\item $ \varphi_3 \not\equiv \varphi_4$, da $ \llbracket \varphi_3 \rrbracket ^I = 1 \neq \llbracket \varphi_4 \rrbracket ^I = 0 \qquad$ mit $ I(A_0) = 0 $; $ I(A_1) = 1 $
		\end{itemize}
		\item Todo	
	\end{compactenum} \
	
	\textbf{Aufgabe 3}
	\begin{compactenum} [(a)]
		\item 
		\begin{compactenum} [(i)]
			\item $ A_{42} $
			\item $ (A_1 \vee \lnot (0 \wedge 1)) $
			\item $ \lnot (0 \wedge A_2) \vee ((\lnot 1 \vee A_5) \wedge (A_3 \vee \lnot ((A_3 \vee \lnot 0) \vee \lnot A_4))) $
		\end{compactenum}
		
		\item $ \varphi $ ist unerfüllbar $\Leftrightarrow \varphi \equiv 0$
		$ \xLeftrightarrow{\text{Dualitätssatz}} \tilde{\varphi} \equiv \tilde{0} \Leftrightarrow \tilde{\varphi} \equiv 1 \xLeftrightarrow{\text{Lemma 2.24b}} \tilde{\varphi} $ ist allgemein gültig.
		\item Todo
	\end{compactenum}\
	
	\textbf{Aufgabe 4}
	\begin{compactenum} [(a)]
		\item Todo
		
		\item Todo
	\end{compactenum}
\end{document}
