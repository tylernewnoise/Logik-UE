\documentclass[a4paper,10pt]{article}
\usepackage[ngerman]{babel}		%dt. Übersetzung und Umlaute
\usepackage[utf8]{inputenc}		%Umlaute direkt eingeben
\usepackage{mathtools}			%Mathekrams
\usepackage{paralist}			%bessere Listen
\usepackage{amssymb}			%Mathesymbole
\usepackage{amsthm}			%typesetting theorems (Text über = u.ä.)
\usepackage{fancyhdr} 			%Headerstyles
\usepackage{verbatim}			%Sourcecode einfügen
\usepackage[margin=2.0cm,headheight=40pt,top=3cm]{geometry}
\usepackage{tikz}
\usepackage{cancel}
\usepackage{stmaryrd}
\usepackage{colortbl}
\usetikzlibrary{matrix,positioning,arrows, automata}

\pagestyle{fancy}
\renewcommand{\headrulewidth}{0.4pt}
\renewcommand{\footrulewidth}{0.4pt}
\lhead{Jonas Panizza (567687)\\
	Fabian Bucher (577513)\\
	Falko Becker (559053)\\
	Blatt 07} %BLATT-NR BEI BEDARF ÄNDERN!
\rhead{ÜG Di 13-15 Jens Keppeler\\
	ÜG Di 13-15 Jens Keppeler \\
	ÜG Di 13-15 Jens Keppeler \\
	Rückgabe ÜG Di 13-15 Jens Keppeler}
\cfoot{}
\rfoot{\thepage}
\begin{document}
	\parindent0pt
	\textbf{Aufgabe 01}
	\begin{compactenum} [(a)]
		\item \begin{compactenum} [(i)]
			\item $ \exists x \exists y \exists z (M(x) \wedge B(y) \wedge F(z)) $
			\item $ \exists x ((Nachfolger(x) = letzter \wedge H(x))\rightarrow H(Nachfolger(x))) $
		\end{compactenum}
		\item \begin{compactenum} [(i)]
			\item Es gibt kein Tag x,  an dem nicht einer der drei Mannschaften 1. FSV Mainz 05, Hertha BSC oder Eintracht Frankfurt Tabellenführer ist = An jedem Spieltag ist einer der drei Mannschaften Tabellenführer.
			\item Eintracht Frankfurt ist nie länger als zwei Tage in Folge Tabellenführer.
		\end{compactenum}
	\end{compactenum}
\end{document}
