\documentclass[a4paper,10pt]{article}
\usepackage[ngerman]{babel}		%dt. Übersetzung und Umlaute
\usepackage[utf8]{inputenc}		%Umlaute direkt eingeben
\usepackage{mathtools}			%Mathekrams
\usepackage{paralist}			%bessere Listen
\usepackage{amssymb}			%Mathesymbole
\usepackage{amsthm}			%typesetting theorems (Text über = u.ä.)
\usepackage{fancyhdr} 			%Headerstyles
\usepackage{verbatim}			%Sourcecode einfügen
\usepackage[margin=2.0cm,headheight=40pt,top=3cm]{geometry}
\usepackage{tikz}
\usepackage{cancel}
\usepackage{mathabx}
\usepackage{stmaryrd}
\usepackage{colortbl}
\usetikzlibrary{matrix,positioning,arrows, automata}

\pagestyle{fancy}
\renewcommand{\headrulewidth}{0.4pt}
\renewcommand{\footrulewidth}{0.4pt}
\lhead{Blatt 08}
\rhead{}
\cfoot{}
\rfoot{\thepage}
\begin{document}
	\parindent0pt
	\textbf{Aufgabe 01}
	\begin{compactenum} [(a)]
		\item \begin{compactenum} [(i)]
			\item $ \varphi(x_{Tel}) = \exists x_K \exists x_A \exists x_{St} (R_{Kino}(x_K, x_A, x_{St}, x_{Tel})\ \wedge \ \exists x_T (R_{Prog}(x_K, x_T, \text{'20:15'})\ \vee \ R_{Prog}(x_K, x_T, \text{'20:30'})))$
			\item $ \varphi(x_R) = \exists x_T \exists x_S ((R_{Film}(x_T, x_R, x_S) \wedge x_R = x_S) \wedge \lnot \forall x_K \forall x_Z R_{Prog}(x_K, x_T, x_Z))$
		\end{compactenum} \
		\item \begin{compactenum} [(i)]
		\item $ \varphi_1(x_T)  $ gibt alle Filmtitel aus, die im Kino Babylon und nicht im Kino Casablanca laufen. 
		\item $ \varphi_2(x_K,x_Z)  $ gibt alle Kombinationen von Kinos und Uhrzeiten aus, an denen ein Film läuft, in dem George Clooney weder Regisseur noch Schauspieler ist.
		\end{compactenum} \
	\end{compactenum} \

	\textbf{Aufgabe 02}
	\begin{compactenum} [(a)]
		\item \begin{compactenum} [(i)]
			\item wahr
			\item wahr
			\item falsch
		\end{compactenum} \
		\item \begin{compactenum} [(i)]
			\item wahr
			\item wahr
			\item falsch
			\item wahr
		\end{compactenum}
		\item \begin{compactenum} [(i)]
			\item 
			\item % Gegenbeispiel: Sei $ \sigma := \{R/1, S/1\} $, wir bretrachten die $ \sigma$-Struktur $ \mathcal{A} := (A, R^\mathcal{A}, S^\mathcal{A}) $ mit $ A:= \{1,2\} $ und $ R^\mathcal{A} = \{1\}; \quad S^\mathcal{A} = \{2\}$. \\
			% Sein nun $ \varphi := R(x) $; $ \psi := S(x) $ \\
			% $ (\exists x R(x) \wedge \exists x S(x)) $
		\end{compactenum}
	\end{compactenum} \

	\textbf{Aufgabe 03}
	\begin{compactenum} [(a)]
		\item
		\item
	\end{compactenum} \

	\textbf{Aufgabe 04}
	\begin{compactenum} [a)]
		\item
		\item \begin{verbatim}
		% (b)
		% voraussetzung
		:- ensure_loaded([al]).

		% wenn die liste L leer ist
		not_member(_, []).

		% wenn X das erste element der liste L ist
		not_member(X, [X|_]) :-
			!, fail.

		% ka
		not_member(X, [_|T]) :- not_member(X, T).

		% (c)
		:- op(300, xfx, <=>). %biimplikation

		% (d)

		% memes
		maybe_sym(F, F)
			:- al_lit(F), !.
		maybe_sym(F, X)
			:- gensym(x, X).

		% dank
		%tseitin(X, S, K) :-

		tseitin(F, [S | T]) :-
			maybe_sym(F, S),
			tseitin(F, S, T).

		tseitin(L, _, []) :-
			al_lit(L), !.
		tseitin(~ F, S, [ S <=> ~ FS | T ] ) :-
			maybe_sym(F, FS),
			tseitin(F, FS, T).

		tseitin(F /\ G, S, [ S <=> (FS /\ GS) | T ] ) :-
			maybe_sym(F, FS),
			maybe_sym(G, GS),
			tseitin(F, FS, FT),
			tseitin(G, GS, GT),
			append(FT, GT, T).

		tseitin(F \/ G, S, [ S <=> (FS \/ GS) | T ] ) :-
			maybe_sym(F, FS),
			maybe_sym(G, GS),
			tseitin(F, FS, FT),
			tseitin(G, GS, GT),
			append(FT, GT, T).

		tseitin(F => G, S, [ S <=> (FS => GS) | T ] ) :-
			maybe_sym(F, FS),
			maybe_sym(G, GS),
			tseitin(F, FS, FT),
			tseitin(G, GS, GT),
			append(FT, GT, T).
		\end{verbatim}
		\item \begin{verbatim}
		\end{verbatim}
		\item \begin{verbatim}
		\end{verbatim}
		\end{compactenum}
\end{document}
