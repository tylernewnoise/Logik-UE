\documentclass[a4paper,10pt]{article}
\usepackage[ngerman]{babel}		%dt. Übersetzung und Umlaute
\usepackage[utf8]{inputenc}		%Umlaute direkt eingeben
\usepackage{mathtools}			%Mathekrams
\usepackage{paralist}			%bessere Listen
\usepackage{amssymb}			%Mathesymbole
\usepackage{amsthm}			%typesetting theorems (Text über = u.ä.)
\usepackage{fancyhdr} 			%Headerstyles
\usepackage{verbatim}			%Sourcecode einfügen
\usepackage[margin=2.0cm,headheight=40pt,top=3cm]{geometry}
\usepackage{tikz}
\usepackage{cancel}
\usepackage{stmaryrd}
\usepackage{colortbl}
\usetikzlibrary{matrix,positioning,arrows, automata}

\pagestyle{fancy}
\renewcommand{\headrulewidth}{0.4pt}
\renewcommand{\footrulewidth}{0.4pt}
\lhead{Jonas Panizza (567687)\\
	Fabian Bucher (577513)\\
	Falko Becker (559053)\\
	Blatt 03}
\rhead{ÜG Di 13-15 Jens Keppeler\\
	ÜG Di 13-15 Jens Keppeler \\
	ÜG Di 13-15 Jens Keppeler \\
	Rückgabe ÜG Di 13-15 Jens Keppeler}
\cfoot{}
\rfoot{\thepage}
\begin{document}
	\parindent0pt
	\textbf{Aufgabe 01} \\
	Todo
	
	\textbf{Aufgabe 02} \\
	\begin{compactenum} [(a)]
		\item 
		\begin{compactenum} [(i)]
			\item $ \lnot A_{73} \in NNF \in DNF \in KNF $
			\item $ (((\lnot A_0 \wedge A_1)\vee 1)\wedge \lnot A_0) \not\in NNF \not\in DNF \not\in KNF$
			\item $ ((\lnot A_4 \wedge A_7)\wedge (A_1 \vee A_1)) \in NNF \not\in DNF \in KNF $
			\item $ (\bigvee\limits_{i = 24}^{42} (\bigvee\limits_{j=37}^{73} (\bigwedge\limits_{k=3}^{4} A_{(i+y) mod\ k}))) \in NNF \in DNF \not\in KNF $
		\end{compactenum}
		\item Todo
	\end{compactenum}\ \\
	\textbf{Aufgabe 04}\
	\begin{compactenum} [(a)]
		\item 
		\begin{verbatim}
		tanz(kreis)
		tanz(hoch(X)) :- tanz(X).
		tanz(runter(X)) :- tanz(X).
		tanz(salto(X)) :- tanz(X).
		\end{verbatim}\
		\item 
		\begin{verbatim}
		gefahr(kreis)
		gefahr(salto(hoch(X))) :- gefahr(X).
		gefahr(runter(hoch(X))) :- gefahr(X).
		gefahr(runter(salto(X))) :- gefahr(salto(X)).
		gefahr(hoch(X)) :- gefahr(X).
		\end{verbatim}
		\item 
		\begin{verbatim}
		duett(kreis, kreis).
		duett(salto(X), salto(Y)) :- duett(X, Y).
		duett(hoch(X), runter(Y)) :- duett(X, Y).
		duett(runter(X), hoch(Y)) :- duett(X, Y).
		\end{verbatim}
	\end{compactenum}
\end{document}
