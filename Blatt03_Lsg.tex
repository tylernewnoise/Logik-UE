\documentclass[a4paper,10pt]{article}
\usepackage[ngerman]{babel}		%dt. Übersetzung und Umlaute
\usepackage[utf8]{inputenc}		%Umlaute direkt eingeben
\usepackage{mathtools}			%Mathekrams
\usepackage{paralist}			%bessere Listen
\usepackage{amssymb}			%Mathesymbole
\usepackage{amsthm}			%typesetting theorems (Text über = u.ä.)
\usepackage{fancyhdr} 			%Headerstyles
\usepackage{verbatim}			%Sourcecode einfügen
\usepackage[margin=2.0cm,headheight=40pt,top=3cm]{geometry}
\usepackage{tikz}
\usepackage{cancel}
\usepackage{stmaryrd}
\usepackage{colortbl}
\usetikzlibrary{matrix,positioning,arrows, automata}

\pagestyle{fancy}
\renewcommand{\headrulewidth}{0.4pt}
\renewcommand{\footrulewidth}{0.4pt}
\lhead{Jonas Panizza (567687)\\
	Fabian Bucher (577513)\\
	Falko Becker (559053)\\
	Blatt 03}
\rhead{ÜG Di 13-15 Jens Keppeler\\
	ÜG Di 13-15 Jens Keppeler \\
	ÜG Di 13-15 Jens Keppeler \\
	Rückgabe ÜG Di 13-15 Jens Keppeler}
\cfoot{}
\rfoot{\thepage}
\begin{document}
	\parindent0pt
	\textbf{Aufgabe 01} \\
	\begin{compactenum} [(a)]
		\item
		\begin{compactitem}
			\item
			\begin{tabbing}
				$ \varphi_{i,j} = $ 
				\= $ ((A_{i,j}\wedge \lnot O_{i,j} \wedge \lnot G_{i,j}\wedge \lnot K_{i,j})\vee(\lnot A_{i,j}\wedge O_{i,j}\wedge\lnot G_{i,j}\wedge \lnot K_{i,j}) $ \\
				\> $ \vee ( \lnot A_{i,j}\wedge \lnot O_{i,j}\wedge G_{i,j}\wedge \lnot K_{i,j})\vee( \lnot A_{i,j}\wedge \lnot O_{i,j}\wedge \lnot G_{i,j}\wedge K_{i,j}))$
			\end{tabbing}
			$ \varPhi_1 := \{\varphi_{i,j} : i,j \in \mathbb{Z}\} $
			
			\item $ \varphi'_{i,j} = (A_{i,j} \rightarrow ((O_{i-1,j} \vee O_{i+1,j} \vee O_{i,j-1} \vee O_{i,j+1}) \wedge (\lnot K_{i-1,j} \wedge \lnot K_{i+1,j} \wedge \lnot K_{i,j-1} \wedge \lnot K_{i,j+1})) $ \\
			$ \varPhi_4 := \{\varphi'_{i,j} : i,j \in \mathbb{Z}\} $
		\end{compactitem} 
		\item $ \varphi''_{i,j} := (K_{i-1,j-1}\vee K_{i-1,j}\vee K_{i-1,j+1}\vee K_{i,j-1}\vee K_{i,j}\vee K_{i,j+1}\vee K_{i+1,j-1}\vee K_{i+1,j}\vee K_{i+1,j+1}) $ \\
		$ \varPhi_2 := \{\varphi''_{i,j} : i,j \in \mathbb{Z}\} $
		\item $ \varphi'''_{i,j} := A_{i,j}\rightarrow(G_{i,j-1}\vee G_{i,j+1}\vee G_{i-1,j}\vee G_{i+1,j}) $ \\
		$ \varPhi_3 := \{\varphi'''_{i,j} : i,j \in \mathbb{Z} \} $
		\item \textcolor{red}{TODO}
	\end{compactenum} \
	
	\textbf{Aufgabe 02}
	\begin{compactenum} [(a)]
		\item 
		\begin{compactenum} [(i)]
			\item $ \lnot A_{73} \in NNF \in DNF \in KNF $
			\item $ (((\lnot A_0 \wedge A_1)\vee 1)\wedge \lnot A_0) \not\in NNF \not\in DNF \not\in KNF$
			\item $ ((\lnot A_4 \wedge A_7)\wedge (A_1 \vee A_1)) \in NNF \not\in DNF \in KNF $
			\item $ (\bigvee\limits_{i = 24}^{42} (\bigvee\limits_{j=37}^{73} (\bigwedge\limits_{k=3}^{4} A_{(i+y) mod\ k}))) \in NNF \in DNF \not\in KNF $
		\end{compactenum} \
		\item \begin{compactitem}
			\item \begin{tabbing}
				$ \varphi := $ 
				\= $ ((\lnot A_3 \vee (A_2 \vee \lnot A_1)) \underline{\wedge} (\lnot A_3 \vee A_0)) \stackrel{\text{Distributivität}}{\equiv} $ \\
				\> $ (((\lnot A_3 \vee (A_2 \vee \lnot A_1))\underline{\wedge} \lnot A_3)\vee ((\lnot A_3 \vee (A_2 \vee \lnot A_1))\underline{\wedge} \lnot A_0)) \stackrel{\text{Distributitivät}}{\equiv}$ \\
				\> $ (((\lnot A_3 \wedge \lnot A_3) \vee ((A_2 \vee \lnot A_1)\underline{\wedge} \lnot A_3))\vee ((\lnot A_3 \wedge A_0)\vee ((A_2 \vee \lnot A_1) \underline{\wedge} A_0))) \stackrel{\text{Distributivität}}{\equiv} $ \\
				\> $ (((\lnot A_3 \wedge \lnot A_3) \vee ((A_2 \wedge \lnot A_3) \vee (\lnot A_1 \wedge \lnot A_3))) \vee ((\lnot A_3 \wedge A_0)\vee((A_2\wedge A_0) \vee (\lnot A_1 \wedge A_0 )))) \in DNF $	
			\end{tabbing}
			\item \begin{tabbing}
				$ \psi :=$
				\= $ (\lnot A_3 \vee ((\lnot A_2 \underline{\rightarrow} \lnot A_1)\wedge A_0)) \stackrel{Eliminiere\ Implikation}{\equiv}$ \\
				\> $ (\lnot A_3 \vee ((\underline{\lnot \lnot} A_2 \vee \lnot A_1)\wedge A_0)) \stackrel{doppelte Negation}{\equiv} $ \\
				\> $ (\lnot A_3 \underline{\vee} ((A_2 \vee \lnot A_1)\wedge A_0)) \stackrel{\text{Distributivität}}{\equiv}$ \\
				\> $ ((\lnot A_3 \vee (A_2 \vee \lnot A_1))\wedge (\lnot A_3 \vee A_0))  \in KNF $
			\end{tabbing}
		\end{compactitem}
		\item \begin{compactitem}
			\item $ \tau_1 = \{\vee, 1\} $ ist nicht adäquat, weil für alle Formeln $ \varphi(X_1,...,X_n) \in AL(\tau_1) $ gilt: $ \varphi[1,...,1] = 1 $ und so die 0 nicht dargestellt werden kann. Beweis Induktion.. \textcolor{red}{TODO} %TODO
			\item $ \tau_2 = \{\lnot, \rightarrow\} $ ist adäquat. Beweis:
			\begin{compactitem}
				\item $ 0 \equiv \lnot (X \rightarrow X) $  für jedes $ X \in AS $
				\item $ 1 \equiv (X \rightarrow X) $ für jedes $ X \in AS $
				\item für alle Formeln $ \varphi , \psi $ gilt:
				\begin{compactitem}
					\item $ (\varphi \vee \psi) \equiv (\lnot \varphi \rightarrow \psi) $
					\item $ (\varphi \wedge \psi) \equiv \lnot (\varphi \rightarrow \lnot \psi) $
				\end{compactitem}
			\end{compactitem}
		\end{compactitem}
	\end{compactenum} \
	\\
	\textbf{Aufgabe 03}
	\begin{compactenum} [(a)]
		\item Die Menge alle Interpretationen $ I $ wird beschrieben durch alle Kombinationen von Abbildungen der Form $ I_1: X_i, Y_i \longmapsto I_1(X_i) = I_1(Y_i) = 1 $ und $ I_2 : X_j, Y_j \longmapsto I_2(X_j) = I_2(Y_j) = 0 $ mit $ i,j \in \{1,...,n\} $ und $ i\neq j$. Da jede Klausel $ (X_i \wedge \lnot Y_i) $ nach der Interpretationsvorschrift $ I_1 $ oder $ I_2 $ abgebildet werden kann und es n verschiedene Klauseln gibt, gibt es $ 2^n $ verschiedene Interpretationen.
		\item \textcolor{red}{TODO}
		\item \textcolor{red}{TODO}
	\end{compactenum}
	\newpage
	\textbf{Aufgabe 04}\
	\begin{compactenum} [(a)]
		\item 
		\begin{verbatim}
		tanz(kreis)
		tanz(hoch(X)) :- tanz(X).
		tanz(runter(X)) :- tanz(X).
		tanz(salto(X)) :- tanz(X).
		\end{verbatim}\
		\item 
		\begin{verbatim}
		gefahr(kreis)
		gefahr(salto(hoch(X))) :- gefahr(X).
		gefahr(runter(hoch(X))) :- gefahr(X).
		gefahr(runter(salto(X))) :- gefahr(salto(X)).
		gefahr(hoch(X)) :- gefahr(X).
		\end{verbatim}
		\item 
		\begin{verbatim}
		duett(kreis, kreis).
		duett(salto(X), salto(Y)) :- duett(X, Y).
		duett(hoch(X), runter(Y)) :- duett(X, Y).
		duett(runter(X), hoch(Y)) :- duett(X, Y).
		\end{verbatim}
	\end{compactenum}
\end{document}
